\input{/Users/daniel/github/config/preamble.sty}
\input{/Users/daniel/github/config/thms-eng.sty}

\begin{document}
\begin{minipage}{\textwidth}
	\begin{minipage}{1\textwidth}
		 \hfill 
		
		{\small \hfill\href{https://github.com/danimalabares/mirror-symmetry}{github.com/danimalabares/mirror-symmetry}}
	\end{minipage}
\end{minipage}\vspace{.2cm}\hrule

\vspace{.5em}
{\LARGE Notes on mirror symmetry}
\tableofcontents
\section{3 may (Alex)}
%Main reference for these notes is \cite{gross}, sect. 14.

We wish to understand
\begin{thm}[Bogomolov-Tian-Todorov]
	Any Calabi-Yau manifold has unobstructed deformations.
\end{thm}

\begin{defn}
	An \textbf{\textit{almost compelx structure}} is an endomorphism $J$…
 \end{defn}
 \begin{remark}
 	It is a fact by Borel \& Serre (1953) that the only spheres which admit an almost complex structure are $S^2$ and $S^6$.
 \end{remark}
\begin{example}
	All complex manifolds are almost complex manifolds.
\end{example}
\begin{thm}
	A necessary and sufficient condition for a $2u$-smooth manifold $M$ to admit an almost complex structure is that the group of tangent bundle of $M$ could be reduced to $\mathsf{U}(n)$.
\end{thm}
\begin{thm}[Newlander-Nirenberg]
	Let $(M,J)$ be an almost complex manifold. Then, the following are equivalent:
	\begin{enumerate}
		\item (six conditions…)
	\end{enumerate}
\end{thm}
\begin{prop}
	An almost complex structure on a real 2-dimensional manifold is a complex structure.
\end{prop}
\begin{proof}
	By the Newlander-Nirenberg theorem, given a point $p\in U\subset M$ and a vector field $\mathfrak{X}{U}$, we have that $(V,JV)$ is a frame, and
	\[N(V,JV)=[V,JV]+J[V,JV]+J[V,J^2V]-[JV,J^2V]=0\]
\end{proof}
\begin{defn}
	A \textbf{\textit{deformation}} of complex analytic space $M$ over a germ $(S,s_0)$ of complex analytic space is a triple $\pi_,X,i)$ such that
	\[\begin{tikzcd}
		X\arrow[d,"\pi",swap]&M\arrow[l,"\text{embedding}",swap]\arrow[d]\\
		(S,s_0)&\operatorname{p t}\arrow[l,"s_0"]
	\end{tikzcd}\]
	where $M$ is a compact manifold, $M\simeq\pi^{-1}(s_0)$ and $\pi$ is proper smooth.
\end{defn}
\begin{thm}[Ehresmann]
	Let $\pi:X\to S$ be a proper family of differentiable manifold. If $S$ is conncected, then all fibres are diffeomorphic.
\end{thm}
\begin{thm}[Kodaira]
	Let $X_0$ be a compact Kähler manifold. If $X\to S$ is a deformation, then any fibre $X_t$ is again Kähler.
\end{thm}
\begin{thm}[Kuranashi]\leavevmode
	\begin{enumerate}
		\item Any compact complex manifold admits a universal deformation.
		\item If $\Gamma(X_0,T_{x_0})=0$ then it admits a universal deformation.
	\end{enumerate}
\end{thm}
\begin{lemma}
	Let $J$ be an almost complex structure sufficiently close to $J_0$ so that it is represented by a form $\lambda\in A^{0,1}T^{1,0}M$. Then $J$ is integrable if and only if 
	\[\bar{\partial}\lambda_i+\frac{1}{2}[\lambda_j,\lambda_j]=0.\]
\end{lemma}
\begin{thm}[Maurer-Cartan]
	\[\bar{\partial}\phi+[\phi,\phi]=0\]
	where
	\[\phi=\phi(t)=\sum_{i=1}\phi_it_i\]
\end{thm}
\begin{defn}\leavevmode
	\begin{itemize}
		\item The \textbf{\textit{Kodaira-Spencer class}} of a one-parameter deformation $J_t$ of a complex stucture $J$ is induced by a homology class $\phi_1\in H^1(X,Tx)$.
		\item The \textbf{\textit{Kodaira-Spancer map}} is
		\begin{align*}
			T_sS&\to H^1(X_s,T_{X_s})=T_{[X_s]}\operatorname{ D e f}(X_{s_0})
		\end{align*}
	\end{itemize}
\end{defn}

\section{May 10}
\subsection{Sergey: preliminaries}
We work in the category of schemes over $\mathbb{C}$.
\begin{defn}
	A morphism $f:X\to Y$ is \textbf{\textit{projective}} if
	\[\begin{tikzcd}[column sep=small]
		X\arrow[rr,"f"]\arrow[dr,hook,"i",swap]&&Y\\
		&\P^n_Y=Y\times\P^n\arrow[ur]
	\end{tikzcd}\]
	where $i$ is a closed embedding and in fact $Y\times\P^n=\operatorname{Spec}(\mathbb{C})$.
\end{defn}
\begin{defn}
	A \textbf{\textit{Hilbert function}} for a given $Z\hookrightarrow\P^N$ is
	\[h_Z(n)=\chi(Z,\mathcal{O}_Z(n))\]
\end{defn}
\begin{defn}[Found later in \cite{huybrechts}, p. 273]
	A morphism $f:(X,\mathcal{O}_X)\to(Y,\mathcal{O}_Y)$ if \textbf{\textit{flat}} if the stalk $\mathcal{O}_{X,x}$ {\color{magenta}[…]}
\end{defn}
\begin{claim}[Criterium for flatness of projective morphisms]
	A projective morhphism is flat if and only if $h_{X_t}(n)$ is constant {\color{magenta}as a function of $t$?
		
	$Y\to\mathbb{Q}[n]$}.
\end{claim}
\begin{example}[non-flat projective morphism (blowup), which is also a non-submersion]
	Let's find some $f:X\to Y$ projective but not flat. Suppose $X,Y$ are smooth and connected.
	
	A closed embedding.
		
		We tried
		\[\begin{tikzcd}[column sep=small]
			&\mathbb{C}^2\times\mathbb{P}^1\arrow[dr]\\
			X=\operatorname{Bl}_0\mathbb{C}\arrow[ur,hook]\arrow[rr,"\pi"]&&Y=\mathbb{C}^2
		\end{tikzcd}\]
		but {\color{magenta}(I think)} its differential is not surjective due to the tangent space of the exceptional divisor.
		
\begin{defn}[\href{https://en.wikipedia.org/wiki/Smooth_morphism}{Wiki}]
	In algebraic geometry, a morphism $f:X\to S$ between schemes is said to be \textbf{\textit{smooth}} if
	\begin{enumerate}
		\item it is locally of finite presentation.
		\item it is flat, and
		\item for every geometric point $\bar{s}\to S$ the fiber $X_{\bar{s}}=X\times_X\bar{s}$.
	\end{enumerate}
\end{defn}
	
	
	
\end{example}
\subsection{Bruno: more on deformation}
\begin{defn}[of smooth submersion] A map whose differential is surjective.
\end{defn}
\begin{defn}[\cite{gross}]
	A \textbf{\textit{deformation}} $X$ consists of a smooth proper morphism $\mathcal{X} \to S$, where $\mathcal{X}$ and $S$ are connected complex spaces, and an isomorphism $X \cong \mathcal{X}_0$, where $0\in S$ is a distinguished point. We call $\mathcal{X}\to S$ a \textbf{\textit{family of complex manifolds}}.
\end{defn}

In order to define the deformation space $\operatorname{Def}(X)$ suppose $X$ is Kähler with $H^0(X,\mathcal{T}_X)=0$. Then there exists a universal deformation:
\begin{defn}[\cite{gross}]
	A deformation $X\to(S,0)$ of $X$ is called \textbf{\textit{universal}} if any other deformation $X' \to (S',0')$ is isomorphic to the pullback under a uniquely determined morphism $\varphi:S'\to S$ with $\varphi(0')=0$.
	\[\begin{tikzcd}
		\mathcal{X}_S\arrow[d,"\pi_S",swap]\arrow[r]&\mathcal{X}\arrow[d]\\
		S\arrow[r,"\exists!",swap]&\operatorname{Def}_S(X)
	\end{tikzcd}\]
\end{defn}
\begin{defn}
	The \textbf{\textit{Teichmüller space}} of $X$ is
	\[\operatorname{Teich}(X)=\frac{\text{complex structures on }M}{\operatorname{Diff}_0}\]
	and it is such that
	\[\mathcal{T}_X\operatorname{Teich}(X)=H^1(X,\mathcal{T}_IX^{1,0})\]
\end{defn}
\begin{remark}[The Misha Verbitsky way]
	Let $X=(M,I)$ and $\bar\partial:C^\infty(M)\to\Omega^1(M,\mathbb{C})$ and remember that
	\begin{itemize}
		\item $\operatorname{img}\bar\partial=\Omega^{0,1}_{(I)}(M)$
		\item $\bar\partial^2=0$.
	\end{itemize}
	Take a solution of the Maurer-Carten equation:
	\[\bar\partial\gamma+[\gamma,\gamma]=0\]
	where $\gamma\in T^{1,0}\otimes\Omega^{0,1}$. Then we do
	\begin{align*}
		(\bar\partial+\gamma)(\bar\partial f+\gamma f)&=\bar\partial(\gamma f)+\gamma\bar\partial f+\gamma(\gamma f)\\
		\bar\partial_{\text{new}}f&=\bar\partial f+\gamma f.\\
		{\color{magenta}…?}
	\end{align*}
\end{remark}
\vspace{2em}
Now take $s\in T^{1,0}\otimes\Omega^{0,1}$ such that
\[\bar\partial s+[s,s]=0\]
and consider also its cohomology class $[s]\in H^1(T^{1,0})$. We have the \textbf{\textit{Kodaira-Spencer map}}
\begin{align*}
	\operatorname{KS}:T_{s_0}S&\to H^1(T^{1,0})\cong T_X \operatorname{Def} X\\
	s&\mapsto [s]
\end{align*}
which is useful because de \textbf{\textit{deformation space}} of $X$ is
\[(\operatorname{Def} X,0)=\frac{\text{solutions to Maurer-Cartan}}{\operatorname{Dif f}_0}\]

Ok, but what is the bracket? Answer: take the usual vector field conmutator on vector fields and the wedge product on differential forms. This makes $(\mathcal{T}^{1,0}_X\otimes \Omega^{0,\bullet}_X,[,],\bar\partial)$ into a \textbf{\textit{differential graded Lie algebra (DGLA)}}.

So suppose
\[s=\sum_{n\geq1}t^ms_m\]
and we wish to find
\[\bar\partial s_1=0\qquad\qquad\text{and}\qquad\qquad \bar\partial s_n=\sum_{i+j=n-1}[s_i,s_j]\]
The right-hand-side equation says $s_n$ is $\bar\partial$-exact.

Now since our objective is to understand Bogomolov-Tian-Todorov, we are interested in what \textbf{\textit{unobsturctedness}} is. It means that
\[\bar\partial s_1=0\qquad\qquad\text{and}\qquad\qquad\bar\partial s_2=[s_1,s_2]\]

Also recall that
\begin{defn}
	Two manifolds $M_1,M_2\subseteq\mathbb{C}^n$ define the same \textbf{\textit{germ}} at $0\in\mathbb{C}^n$ if there is an open set $U\subseteq \mathbb{C}^n$ containing $0$ such that
	\[M_1\cap U=M_\cap U.\]
\end{defn}
and then…
\begin{thm}[Bogomolov-Tian-Todorov]
	content...
\end{thm}

\subsection{Griffiths transversality (Victor)}
\begin{claim}
	Let $X$ be a complex manifold. For a 1-parameter family of complex structures $(X,J_t)$ and forms $\alpha_t\in\Omega^{p,q}(X,J_T)$ we have
\[\frac{d}{dt}\Big|_{t=0}\alpha_t\in\Omega^{p+1,q-1}(X)\oplus\Omega^{p,q}(X)\oplus\Omega^{p-1,q+1}(X).\]
\end{claim}
\begin{proof}
	{\color{magenta}content...}
\end{proof}

\subsection{Hodge Theory for Calabi-Yau (afternoon)}
\subsubsection{Preliminaries (Sergey)}
Let's first recall that
\begin{defn}
	The \textbf{\textit{Hodge star}} operator is 
	\begin{align*}
		*:H_\partial^{p,q}&\to H_{\bar\partial}^{n-q,n-p}\\
	\end{align*}
\end{defn}
\begin{prop}
	For any complex manifold,
	\begin{align*}
		H^{p,q}_{\bar\partial}\times H^{n-p,n-q}_{\bar\partial}&\to H^{n,n}_{\bar\partial}\cong\mathbb{C}\\
		([\alpha],[\beta])&\mapsto\int_{[X]}\alpha\wedge\beta:=(\alpha,\beta)
	\end{align*}
	is bilinear and non-degenerate.
\end{prop}
\begin{proof}
	$\forall\alpha\exists\beta=*\bar\alpha$ such that $(\alpha,\beta)\neq0$ so
	\begin{align*}
		0<\|\alpha\|^2=\int\alpha\wedge*\bar\alpha\\
		\langle\alpha,\beta\rangle=\int\alpha\wedge*\bar\beta
	\end{align*}
	{\color{magenta}where $\langle-,-\rangle$ is the induced metric by some hermitian/riemannian metric on $X$?}
\end{proof}

\begin{thm}[Serre duality]
	For any complex manifold,
	\begin{align*}
		H^{p,q}_{\bar\partial}\times H^{n-p,n-q}_{\bar\partial}&\to H^{n,n}_{\bar\partial}\cong\mathbb{C}\\
		([\alpha],[\beta])&\mapsto\int_{[X]}\alpha\wedge\beta
	\end{align*}
	is a perfect pairing. That is
	\[H^{p,q}\cong(H^{n-p,n-q})^*\]
\end{thm}
And we also have
\begin{thm}[Hodge]
	For Kähler manifolds
	\[H^{p,q}\cong \overline{H^{q,p}}\]
\end{thm}
\subsubsection{Pseudoholomorphic curves (Victor)}
We follow \cite{aroux}, lecture 3.
\begin{remark}
	For every Calabi-Yau manifold $X$,
	\[H^{p,0}= H^{n,n-p}=H^{n-p}_{\bar\partial}(X,\Omega^n_X)=H_{\bar\partial}^{n-p}(X,\Omega_X)=H^{0,n-p}=H^{n-p,0}\]
\end{remark}
So we have some symmetry:
\[\begin{tikzcd}[column sep=tiny,row sep=tiny]
	&&&1&&&&\\
	&&0&&0&&\\
	&0&&a&&0\\
	1&&b&&b&&1\\
	&0&&a&&0\\
	&&0&&0\\
	&&&1
\end{tikzcd}\qquad\qquad
\begin{tikzcd}[column sep=tiny,row sep=tiny]
	&&&1&&&&\\
	&&0&&0&&\\
	&0&&b&&0\\
	1&&a&&a&&1\\
	&0&&b&&0\\
	&&0&&0\\
	&&&1
\end{tikzcd}\]
\begin{defn}
	Let $(X^{2n},\omega)$ be a symplectic manifold, $J$ a compatible almost-complex structure, $\omega(\cdot,J\cdot)$ the associated Riemannian metric. Furthermore, let $(\Sigma, j)$ be a Riemann surface of genus $g$ and $z_1,\ldots,z_k$ marked points.

	There is a well-defined moduli space of $\mathcal{M}_{g,k}=\{(\Sigma,z_1,\ldots,z_k)\}$ which is a complex manifold of dimension $3k-3+k$.



	$u:\Sigma\to X$ is a \textbf{\textit{$J$-holomorphic (or pseudoholomorphic) map}} if
	\[J\circ du=du\circ j\]
	that is,
	\begin{equation}\label{eq:CR}
		\bar\partial_Ju=\frac{1}{2}(du+Jduj)=0.
	\end{equation}
	For $\beta\in H_2(X,\mathbb{Z})$, we obtain an associated space
	\[M_{g,k}(X,J,\beta)=\{(\Sigma,j,z_1,\ldots,z_k,u:\Sigma\to X|u_*[\Sigma]=\beta,\bar\partial_Ju=0\}/\sim\]
	where $\sim$ is the equivalence given by $\phi$ below:
	\[\begin{tikzcd}
		\Sigma,z_1,\ldots,z_k\arrow[r,"u"]\arrow[d,"\phi",swap]\arrow[d,"\cong"]&X\\
		\Sigma',z_1',\ldots,z_k'\arrow[ur,"u'",swap]
	\end{tikzcd}\]
\end{defn}
\begin{question}
	Where does the object in \cref{eq:CR} live? The differential of any map of complex manifolds can be decomposed in $\partial$ and $\bar\partial$. The operator $\bar\partial_Ju$ is an element of $\Omega^{0,1}(\Sigma,u^* TX)=\Gamma(\Sigma,\Omega^{0,1}(\Sigma)\otimes u^*TX)$.
\end{question}
\begin{remark}
	See \href{https://en.wikipedia.org/wiki/Pseudoholomorphic_curve#Analogy_with_the_classical_Cauchy–Riemann_equations}{wiki} for interpretation of this definition as a map satisfying the Cauchy-Riemann equations.
\end{remark}
\begin{remark}
	See \href{http://members.unine.ch/felix.schlenk/Maths/What/pseudoholomorphic.pdf}{What is... a pseudoholomorphic curve?} for another friendly explanation:
	\begin{quotation}
		A pseudoholomorphic curve is just the natural modification of the notion of a holomorphic curve to the case when the ambient manifold is almost-complex.
	\end{quotation}
\end{remark}

\section{May 17}
\subsection{Pseudoholomorphic curves cont. (Victor)}
We continue to read \cite{aroux}, lecture 3.
\begin{defn}
	We say that $u:\Sigma\to X$ is \textbf{\textit{simple}} if there exists $z\in\Sigma$ such that $du(z)\neq0$ and $u^{-1}(u(z))={z}$.
\end{defn}
Which roughly means that the function is not generically one to one on its image.
\begin{example}
	The function
	\begin{align*}
		u:\mathbb{P}^1&\to\mathbb{P}^2\\
		[x:y]&\mapsto[x^2:y^2:0]
	\end{align*}
	is not simple. Indeed, near a point $[x:y]\in\P^1$ with $x\neq0$, the differential of $u$ may be expressed in coordinates as the linear map $du=\begin{pmatrix}
		2&0
	\end{pmatrix}\neq0$; however $u^{-1}([x^2:y^2:0])=\{[x:y],[-x:y]\}$. The case of $y\neq0$ is analogous. We also see there are no singular points, so $u$ cannot be simple.
\end{example}
Then we define
\begin{align*}
	D_{\bar\partial}:W^{r+1,p}(\Sigma,u^*TX)\times T\mathcal{M}_{g,k}\to W^{r,p}(\Sigma,\Omega^{0,1}_\Sigma\otimes U^*TX)
\end{align*}
by 
\[D_{\bar\partial}(v,j')=\bar\partial v+\frac{1}{2}(\nabla_vJ)du\cdot j+\frac{1}{2}J\cdot du\cdot j'\]
where $W^{r,p}$ is a completion of $C^\infty(-)$ of (?) to $L^{r,p}$ norm defined by $\|f\|_{r,p}=\left(\sum_{i=0}^r\int|f^{(i)}(t)|^pdt\right)^{1/p}$.

$D_{\bar\partial}$ is Fredholm, (meaning the dimensions of its kernel and cokernel are finite), with index (the difference of such numbers)
\[\index_\mathbb{R} D_{\bar\partial}:=2d=2\langle c_1(TX),\beta\rangle+n(2-2g)+(6g-6+2k).\]

We may interpret this equation as differentiation of the Cauchy-Riemann equations.

\subsection{Dirichlet energy functional (Alex)}
	We follow \cite{mcduff}, sec. 2.2
	
	Consider a map
	\[u:(\Sigma,j)\to(X,\omega,J,g)\]
	and define the \textbf{\textit{energy functional}}
	\[\varepsilon=\int_\Sigma|du|^2_g\operatorname{Vol}_g\]
	Now, we may take local isothermic coordinates where the metric is expressed as
	\[g=\lambda(x,y)(dx^2+dy^2)\]
	giving
	\begin{align*}
		du&=\partial_xu\otimes dx+\partial_yu\otimes dy\\
		|du|^2&=|\partial_xu|^2\lambda^{-2}+|\partial_yu|^2\lambda^{/2}\\
		\operatorname{Vol}_{\Sigma g'}&=\lambda^2dx\wedge dy
	\end{align*}
	Then
	\[\varepsilon(u)=\int_\Sigma|\partial_xu|^2_g+|\partial_yu|_g^2dx\wedge dy.\]
	The following equality shows that the energy functional attains its minimum on pseudoholomorphic maps (in virtue of \cref{eq:CR}).
	\begin{claim}
		For every smooth map $u:\Sigma\to X$,
	\[\varepsilon(u)=\int_\Sigma2|\bar\partial_J|^2\operatorname{Vol}+\int_\Sigma u^*\omega\]
	\end{claim}
	\begin{proof}
		content...
	\end{proof}

\section{May 24 (Alex)}
We start with two short questions from last session.
\begin{question}\leavevmode
	\begin{itemize}
		\item What exactly is $\Omega^1(\Sigma,E)$ where $E$ is a vector bundle? It is the space of sections of the bundle $T^*M\Sigma\otimes E$.
		
		\item Let $u:(\Sigma,j)\to(X,J)$. Is $du$ is an element of $\Omega^{1}(\Sigma,u^*TM)$? Yes, notice that $du$ is an element of $\operatorname{Hom}(T\Sigma,TX)$. Forget about all of $TX$ and consider only its image under $u$. There is an isomorphism $T\Sigma^*\otimes u^*TX\cong \operatorname{Hom}(T\Sigma,u^*TX)$.
	\end{itemize}
\end{question}
\begin{remark}
	There is a bundle $\mathcal{E}\to\mathcal{B}$ where $\mathcal{B}=C^\infty(\Sigma,M)$ and the fibers are $\mathcal{E}_u=\Omega^{0,1}(\Sigma,u^*TM)$. For a map $u:(\Sigma,j)\to(X,J)$, the nonlinear operator
	\[u\mapsto(u,\bar\partial_J(u))\]
	is a sections of this bundle whose zero-set is the space of $J$-holomorphic curves.
\end{remark}
Then we concluded the proof of the final claim of the last session:

\begin{lemma}[2.2.1, \cite{mcduff}]
	Let $\omega$ be a nondegenerate 2-form on a smooth manifold $M$. If $J$ is $\omega$-tame then every $J$-homolomorphic curve $u:\Sigma\to M$ satisfies the energy identity
	\[E(u)=\int_\Sigma u^*\omega.\]
	If $J$ is $\omega$-compatible then every smooth map $u:\Sigma\to M$ satisfies
	\[E(u)=\int_\Sigma|\bar\partial_J(u)|^2_J\operatorname{Vol}_\Sigma+\int_\Sigma u^*\omega.\]
\end{lemma}
\begin{proof}
	We may take local isothermic coordinates where the metric is expressed as
	\[g=\lambda(x,y)(dx^2+dy^2)\]
	giving
	\begin{align*}
		du&=\partial_xu\otimes dx+\partial_yu\otimes dy\\
		\implies |du|^2&=|\partial_xu|^2\lambda^{-2}+|\partial_yu|^2\lambda^{/2}\\
		\operatorname{Vol}_{\Sigma g'}&=\lambda^2dx\wedge dy
	\end{align*}
	Then
	\[\varepsilon(u)=\int_\Sigma|\partial_xu|^2_g+|\partial_yu|_g^2dx\wedge dy.\]
\end{proof}

\section{June 7 Frobenius manifolds (Sergey)}%
\label{sec:June 7}

%Recall from \cite{dubrovin} that he invented Frobenius manifolds as the coordinate-free form of WDVV. What is WDVV?

\subsection{Introduction}
Dubrovin in 1991 formulated the notion of \textit{\textbf{Frobenius Manifold}} in the context of the \textit{\textbf{WDVV equation}} in singularity theory. In late 1970s or early 1980s, Kyoji Saito formulated the notion of \textit{\textbf{flat structure}}, or \textit{\textbf{Saito (pre-)structure}}. In 1962, when May (?) was 24, there was a lot of activity in Paris. Not far from then in Japan, Saito was looking for a PhD, and eventually became a student of Brieskorn (though he initially intended to work with Grauert).

Saito started studiyng quotient singularities, and then we continued to all isolated singularitied.

\textit{\textbf{Brieskorn sphere}} is given by
\[\begin{cases}
x_1^2+x_2^2+x_3^2+x_4^2+x_5=0\\
\sum|x_i|^2=\Sigma
\end{cases}\]
in $\mathbb{C}^5$.

More generally we can study links of \textit{\textbf{isolated singularities}}. Let $f\in\mathbb{C}[[x_1,\ldots,x_n]]$ and define $H=\{f(x)=0\}$.

What is $T_0^*H$? Well,
\[T_0^*H=\frac{\mathfrak{m}}{\mathfrak{m}_0}\] where $\mathfrak{m_0}=\langle x_1,\ldots,x_n\rangle$ is a maximal ideal.

The important thing is $f(0)=0\iff f\in\mathfrak{m}_0$. Also, $0\in H_{\operatorname{ s i n g}}\iff f\in \mathfrak{m}_0^2$.

\begin{defn}
	Given $\mathbb{C}^*=V$, ie. a $\mathbb{Z}$-grading of $V$, then $f\in S^\bullet V^*$ is called \textit{\textbf{quasi-homogeneous}} of degree $D$ and weight $(w_1,\ldots,w_n)$ if
	\[\lambda^*f=\lambda^D\cdot f\]
that is,
\[f(\lambda^{w_1}x_1,\lambda^{w_2}x_2,\ldots,\lambda^{w_n}x_n) =\lambda^Df(x_1,\ldots,x_n)\]
\end{defn}
\begin{prop}[Euler]
	The \textit{\textbf{Euler vector field}} is $\sum x_i \frac{\partial }{\partial x_i}$. So,
	%\begin{equation*}
	%	E(f)=\sum \frac{\partial f}{\partial x_i}=\sum x_i \frac{w_if}{x_i}=\sum w_if=Df
	%	\label{par:}
	%	
	%\end{equation*}
\end{prop}
\begin{proof}
	$\lambda\to 1+\varepsilon \ldots$
\end{proof}

Another way of saying that $f$ is quasi-homogeneous is that $f$ is eigenvector of $E_w$ with eigenvalue $D$.

\begin{exercise}
	Homogeneous singularity is isolated iff $Z(f)\subset \mathbb{P}(V)$ is smooth.
\end{exercise}

If you have weights, you have \textit{\textbf{weighted projective space}}: $\mathbb{P}(w_1,\ldots,w_n)=\operatorname{Proj} k[x_1,\ldots,x_n]$ for $x_i$ of degree $w_i$.

\begin{defn}
	$w$ is \textit{\textbf{well-formed}} if $\forall i$, $\operatorname{grd}(w_1,w_2,\ldots,\hat{w_i},\ldots,w_n)=1$.
\end{defn}

Look for some text containing "singular locus of weighted projective space".

\begin{example}
	$\mathbb{P}(1,\ldots,1,d)$. So we have $\mathbb{P}(1,1,d)\overset{\mathcal{O}(d)}{\hookrightarrow} \mathbb{P}^{d+1}$, which is the Vernoese embedding. Secretely $\mathbb{P}(1,1,d)$ is a projective cone over $v_d(\mathbb{P}^1$ with singularities its vertex.

	We can resolve this singularity by blowing up: $Y\overset{\mathcal{L}}{\hookrightarrow}\mathbb{P}^N$, $\operatorname{ B l}_p\mathbb{P}$.
\end{example}

\subsection{Milnor ring and Milnor number}
$f$ isolated singularities iff
\begin{equation*}
	\dim\mathbb{C}[[x]]/\left( \frac{\partial f}{\partial x_i} ,f \right) <\infty
\end{equation*}
Define the latter ring to be the \textbf{\textit{Milnor ring}}:
\[\mu_{\operatorname{ e v}}=\mathbb{C}[[x]]/\left( \frac{\partial f}{\partial x_i} ,f \right)\]

There's also the \textbf{\textit{Milnor fibration}}, and the \textbf{\textit{Milnor map}}:
\[\begin{tikzcd}
	S_\varepsilon\arrow[dr,"\text{Milnor map}",swap]\arrow[r]&\mathbb{C}^*\arrow[d,"\arg"]\\
	&\mathsf{U}(1)\cong S^1
\end{tikzcd}\]
\begin{thm}[Milnor]
	$\mu_{\operatorname{ a l g}}=\mu_{\operatorname{ to r}}$
\end{thm}

\subsection{Versal deformation of isolated singularities}
Let $\lambda\in\mathbb{C}^n$,
\[f_\lambda = f+\lambda_1g_1+\lambda_2g_2+\ldots+\lambda_\mu g_\mu\]

\begin{thm}[Saito]
	On versal deformation spaces of isolated singularities there exists a rich structure called \textbf{\textit{flat structure}}, a \textbf{\textit{primitive form}}, and some higher pairings, and so on and so on.
\end{thm}

So, Dubrrovin defined Frobenius manifolds in 1991, and what is it?
\begin{defn}
	Take a manifold either real or complex, and introduce some geometric structure as follows: a metric $g$, a $3$-tensor $C$, such that $\forall m\in M$, $T_mM$ has the structure of a Frobenius algebra.
\end{defn}

\subsection{Frobenius algebra}
You have $A$ and a pairing $A\otimes A\to A$. So the 3-tensor is
\begin{align*}
	g(a\cdot b,c)=g(a,b\cdot c)
\end{align*}
You can also write this as $g(e,a)=\tau(a)$ for neutral element $e$. Notice this is the same as 2d TQFT. Also a connection
\[\nabla =\nabla^g+K\]
and we can also consider a 1-parametric family of connections
\[\nabla^{(\alpha)} =\nabla^g+\alpha K\]
with $\nabla^{(\alpha)}$ flat for all $\alpha$. So this is a pencil of metrics.

\begin{quotation}
	For every isolated singularity, its space of versal deformations has a Frobenius manifolds structure.
\end{quotation}
\section{Once upon a time in semester 2024.1: Fano manifolds and Mirror symmetry}

\subsection*{Fano manifolds}
	\begin{itemize}
		\item Algebraic geometry
		
		Let $X$ be a compact closed complex submanifold of $\mathbb{C}\mathbb{P}^N$.
		\[X\hookrightarrow \mathbb{C}\mathbb{P}^N\]
		Also, $H=\mathbb{C}\mathbb{P}^{N-1}$ is a submanifold of $\mathbb{C}\mathbb{P}^N$ so consider the intersection $X\cap H$. Also we have an ample bundle,
		\[c_1(i^*(\mathcal{O}(1)))=H^1\sim c_1(X)=c_1(Tx)\]
		
		\item Differential geometry
		
		Also we have a symplectic form $\omega$ such that
		\begin{enumerate}
			\item $[\omega]=c_1(X)\in H^2(X,\mathbb{Z})$.
			\item $\omega$-Kähler, so $d\omega=0$ and $\omega(u,Ju)>0$ for all $u\neq 0$ and $\omega>0$.
		\end{enumerate}
		Also we have that $\operatorname{H o l}_g\subset \mathsf{U}(N)$.
		
		\item Symplectic geometry
		
		We have that $\omega$ is compatible with the complex structure $J$ and also we have a Riemannian metric. Let's define a manifold $M$ to be \textbf{\textit{monotone}} if $c_1(M,\omega)=\lambda[\omega]$ for some $\lambda\in\mathbb{R}$ positive. This is related to the ample line bundle.
		
		This is a \textit{rigid object}. (So we expect to find something rigid on the other side of the mirror.)
	\end{itemize}
	\begin{example}\leavevmode
		\begin{enumerate}
			\item A point.
			\item Product of Fano manifolds.
			\item $\mathbb{P}^1\subset\mathbb{A}$ given by $x=y^{-1}$.
			\item $\frac{dx}{x}=-\frac{dy}{y}\implies \omega^*_x=\mathcal{O}(+2)$.
			\item $\mathbb{P}^n$.
			\item Hypersurface of degree $d$, $X_d\subset\mathbb{P}^n$. Solutions of polynomials.
			\begin{itemize}
				\item Degree 1 is just hypersurfaces.
				\item Degree 2, quadric.
				\item Degree 3, cubic. Famous work of1972 Griffiths and Clamens. They invented the technique of intermediate jacobians, …
				\item Degree 4. They are birational. Iskovskisk-Manin.
			\end{itemize}
			\item Grassmanians $\operatorname{Gr}(k,N)$. Is an example of $G/P$, P for parabolic and it contains Borel.
			\item $\operatorname{F l}(1,2,\ldots,N)$, so $W_1\subset W_2\subset\ldots W_N$, which is an example of $G/B$, B for Borel.
			\item Fano manifolds are very important in Minimal Model Problem (MMP).
		\end{enumerate}
	\end{example}
	
	Properties of a Fano manifold $F$:
	\begin{enumerate}
		\item $\pi_1(F)=0$.
		\item $H^i(F,\mathcal{O}_F)=H^k(F,\omega\otimes\omega^\wedge=0$. This implies that Todd genus $\chi(F,\mathcal{O})=1=\sum(-1)\dim H^k(F,\mathcal{O})=\int_{[x]}\operatorname{T d} X$ and the covering $\tilde{D}\overset{d}{\to} F$ is also a fano manifold.
		\item All this means that hodge numbers $h^{k,0}=h^{0,k}=0$.
		\item $\operatorname{Pic} F\cong \mathbb{Z}^\rho$ where $\rho=b_2$, so it is torsion free.
		\[\begin{tikzcd}
			0\arrow[r]&\operatorname{Pic} F\arrow[r,"c_1"]&H^2(F,\mathbb{Z})\arrow[r]&0
		\end{tikzcd}\]
		\item Now consider the open complex cone $\mathcal{K}$ of a Kähler manifold, it is a complicated object. For Fano, this cone is rational polyhedral---this follows from famous cone theorem: there are two duals, second cohomology, $H^2(X,\mathbb{R})=\mathbb{R}^\rho$, and second homology, $H_2(X,\mathbb{R})$, so if you have a complex curve $C\subset X$ its class is $[C]\in N_1$. Look for curves that $c_1>0$, so $(-\mathcal{K} x)\cdot c>0$ so on the side of cohomology you have the K\"hler cone of $\mathcal{K}$ of ample classes  and on the side homology there's the Mory cone $\mathcal{M}=\langle[c]\rangle$ and it turns out that the two cones are dual. So one is polyhedral if the other is.
		
		[…] then for every ray there is a map $F\to F_\rho$ such that it contracts all the rays in this class and only them, universal property.
		
		If you have any space $Y$, there exists so-called MRC fibration so that spaces are \textbf{\textit{rationally connected}}: for every two points you can draw rational curves. For example, for the projective space we get a point on the base.
		
		\item $F$ is arc-rationally connected. Of course, it is path-connected. Theorem by KKM, 1990.
		
		\item Deformation of Fano manifold is Fano due to positivity of the bundle (?).
		
		\item $\Gamma(X,\Omega^{\otimes N})=0$.
	\end{enumerate}
	
	\begin{thm}[Very important, KMM]
		For any dimension $d\geq0$ there exist only finitely many deformation classes of Fano manifolds.
	\end{thm}
	This yields:
	\begin{itemize}
		\item $d=0$, point
		\item $d=1$, $\mathbb{P}^1$.
		\item $d=2$, the are 10 \textbf{\textit{del Pezzo}} surfaces.
		\begin{itemize}
			\item $\mathbb{P}^2$.
			\item $\mathbb{P}^1\times \mathbb{P}^1$ (the only one that is not a blow up.)
			\item $\operatorname{B l}_{P,Q}\mathbb{P}^2$.
			\item $\operatorname{B l}_{P_1,P_2,P_3}\mathbb{P}^2$.
			Let's introduce some funny number: $\int_{[X]}c_1(X)^n=d>0$ the \textbf{\textit{degree}}, it is bounded by some number $F(d)$. What is \textbf{\textit{degree of $\P^n$}}? $(-K_{\mathbb{P}^n})^n=(n+1)^n$.
			
			\textbf{\textit{Noether's formula}} says that 
			\[d+(\rho+2)=c_1^2+c_2=(2\chi(\mathcal{O}))\]
			so
			\[d+\rho=10\]
			And we can bound the degree: $1\leq d\leq 9$. We have
			\[\begin{tikzcd}
				9\arrow[r,no head]&8\arrow[r,no head]&7\arrow[ld,no head]\arrow[r,no head]&6\arrow[r,no head]&5\arrow[r,no head]&4\arrow[r,no head]&3\arrow[r,no head]&2\arrow[r,no head]&1\\
				&8
			\end{tikzcd}\]
			So this reminds us of the Dynkin diagram of $E_{10}$. Look at degree 3. It's a cubic surface $X_3\subset\mathbb{P}^3$. It has 27 lines and 27 pencils of conics and corresponds with branes and compactifications of ?. Also $E_6$ has 27 something and it's not a coincidence. In fact Cartan defined $E_6$ as the automorphisms of… using that there's 45 planes, intersecting… ?
			
			Consider double cover of $\mathbb{P}^2$.
			
			Also $E_7·$ is related to degree 2, they have 56 lines on them. $E_8$ to degree 1, it is very complicated of course so there's 248 lines on it.
			
			Up to degree 6 they are toric surfaces. 5 already is rigid and it has no moduli, finite group of automorphisms, symmetric group acts here.
			
			Du Val: ADE singularities.
			
			5 is $\overline{\mathcal{M}_{0,5}}$ its a moduli space. It's nice.
			
			Another property is $H^{1,1}=H^2(X,\mathbb{R})$. There is a unimodular Lorenzian lattice $(H^2(X,\mathbb{Z}),\cap\text{ pairing})$.
		\end{itemize}
		\item $d=3$ there are 105. $b_2\to 17$ Poincaré missed it.
		\item $d=4$ nobody knows at all.
	\end{itemize}
	
	\subsection*{Mirror symetry}
	Corresponding to Fano manifolds, we have \textbf{\textit{Landau-Ginzburg models}}, which are pairs $(Y,W)$ where $Y$ is a quasi-projective variety and $W\in\Gamma(Y,\mathcal{O}_Y)$. Fantastic Laurent Polynomials.
	
	\subsection{The conifold point}
	This is the abstract of \href{http://www.mpim-bonn.mpg.de/node/5969}{a talk by Sergey in 2015 at Bonn}:
	\begin{quotation}
		Conjecture $O$ describes the geometry in the complex line of the eigenvalues $u_i$ of the operator of quantum multiplication by the first Chern class acting on the cohomology of a Fano manifold. In particular, it says that eigenvalues with maximal absolute value have multiplicity one and one of them is real and positive number $T$.
		
		Fano manifolds tend to have mirror dual Ginzburg-Landau potentials $f$, which
		tend to have a distinguished non-degenerate critical point which we name the
		\textbf{\textit{conifold point}}. Explicitly the conifold point is the unique critical point $P$ on real positive locus, and the respective critical value $T_{\operatorname{ co n}} = f(P)$ is the global minimum on the real positive locus. \textbf{In this case it is conjectured that $T_{\operatorname{ co n}}$ conicides with $T$}, that is any eigenvalue $u_i$ has absolute value at most $T_{\operatorname{ co n}}$, and that the conifold point is the unique critical point with value $T_{\operatorname{ co n}}$.
		
		In most cases the existence of the conifold point is clear and the conjectures
		can be checked to be true, however we do not know how to prove them even
		for abstract toric Fano manifolds, or complete intersections therein. These conjectures
		are basic for formulation of Gamma conjectures about the appearance of Gamma
		unction in the symplectic topology of Fano manifolds. Two references are arXiv:1404.7388
		and my joint work with Golyshev and Iritani arXiv:1404:6407.
	\end{quotation}
	Also see who quoted \textit{The conifold point}.  \href{https://scholar.google.com/scholar?start=20&hl=es&as_sdt=2005&sciodt=0,5&cites=11808564544162580683&scipsc=}{here}. And also don't forget that \href{https://en.wikipedia.org/wiki/Combinatorial_mirror_symmetry}{wiki on combinatorial mirror symmetry} exists!
	
	\subsection{the history of quantum cohomology}
	\paragraph{-1.} Before mirror symmetry, in 1985, people tried to classified vector bundles in $\mathbb{C} P^2$. There are \textbf{\textit{rigid bundles}}.  (See J.M. Drezet et J. Le Pother.) So in 1987 Garolov and ? (she studied at the same time as Kontsevich) called them \textbf{\textit{Markov equations}} of the form $x^2+y^2+z^2=3xys$.
	
	So, there are three independant morphisms $(\mathcal{O}(0),\mathcal{O}(1),\mathcal{O}(2))\rightsquigarrow?$, difficult to specify formally. So people introduced simmer orb decompositions, and then introduced notion of Serre functor, which is donde by consider two bundles and $R\operatorname{Hom}(F,S\Sigma)$ and it is just the twist canonical bundle.
	
	Then some people introduced to $A_\infty$ algebras and $L_\infty$ algebras. $A_\infty$ algebra is very important because it is decomposed as $A=A_{\operatorname{ e v}}\oplus A_{\operatorname{ o d d}}$.

	Some other old friends like Penner, who gave a cell decomposition of the moduli space $\mathcal{M}_{g,n}\times\mathbb{R}^n$. And Kontsevich actually uses this on his 90's book on the Witten conjectures. Then you can get some graph cohomology out of this. Kontsevich went to Harvard in the 90's and gave two lectures.
	
	So in the 90's physical mathematics really exploded.
	
	\begin{conjecture}[Witten]
		\[\int_{[\bar{\mathcal{M}}_{g,n,n}]}\bigcap KdV\]
	\end{conjecture}
	
	And then also Vafa with some \textbf{\textit{quantum rings}} with multiplicities on $H^\bullet(X)$ and $X$ K\"ahler manifold taget in 2d $N=(2,2)$ QFT. \textbf{\textit{White's associative product}}. But why is it associative? Beginings of cohomology.
	
	In winter 92 Fukaya gave a lecture where he related some theories in codimensions and then a category, symplectic manifold $(Y,\omega)$. Take Lagrangian manifolds as objects and then 
	
\section{November 26: Fano 3-folds}

Take $F\subset \mathbb{P}^{g+2}$ a projective variety so that $C=F \cap\mathbb{P}^g$ is  canonically embedded curve.

\begin{defn}\leavevmode
	\textit{\textbf{Del Pezzo Variety}} is a Fano $d$-fold of degree $d-1$.
\end{defn}

\begin{defn}[Iskovskrkm-Manin]\leavevmode
	Fano variety $\to $ projective, $-k>0$ (ample).
\end{defn}

\begin{remark}\leavevmode
	Often Fano variety $\to$ $\mathbb{Q}$-Goresnstein singularities, $-h K_X$ is ample Cartier divisor.
\end{remark}

\begin{question}\leavevmode
	What is the relationship between Fano varieties and positive curvature?
\end{question}

There was a discussion about \textit{\textbf{varieties of sums of powers (VSP)}} which are
 \begin{align*}
\operatorname{ V SP}(f_d,k)&=\left\{ \ell_1,\ldots,\ell_p \in \mathbb{P}( V^*):\exists \alpha_1,\ldots,\alpha_k \in \mathbb{C},\;  \right\} 
\end{align*}

Again:

{\color{3}\bfseries Algebraic geometry.}\hspace{.5em} $-K>0$.

 {\color{5}\bfseries Differential/Riemannian geometry.}\hspace{.5em} $\operatorname{Ric}_g \geq 0$ Kähler compact.

 {\color{2}\bfseries Symplectic geometry.}\hspace{.5em} $(M,\omega)$ positive monotone symplectic manifold, $c_1(M,\omega)=\lambda[\omega]$, $\lambda>0$. $[\omega] = c_1(M,\omega)\in H^{2}(M, \mathbb{Z})$

From DG to AG just use our knowledge from complex geometry. The other way is a deep theorem by Yau of PDE.

Deep theorem of Taukes shows that the purely symplectic setting is in fact algebraic:

\begin{thm}[Taukes]\leavevmode
$\operatorname{Gr}=\operatorname{SW}$ where $\operatorname{Gr}$ is Gromov and $\operatorname{SW}$ is Seiberg-Witten invariant (that's another PDE).
\end{thm}

\begin{coro}\leavevmode
	All symplectic Fano manifolds in dimension 4 are algebraic.
\end{coro}

We also went over the same diagram here
\[\begin{tikzcd}
				9\arrow[r,no head]&8\arrow[r,no head]&7\arrow[ld,no head]\arrow[r,no head]&6\arrow[r,no head]&5\arrow[r,no head]&4\arrow[r,no head]&3\arrow[r,no head]&2\arrow[r,no head]&1\\
				&8
			\end{tikzcd}\]
But now we put $S_i$ instead of just a number. I think it's because  $S_i$ are surfaces of degree $i$.

{\color{3}\bfseries Manin cubic forms.}\hspace{.5em} Because it's fano and so on, we have that $H^{2}(S,\mathbb{Z})=\operatorname{Pic}(S)$. Its a $(1,\rho-1)$ lattice. 

\begin{exercise}\leavevmode
	$-K_S$ is a vector of self intersection  $(2,2)$. Show that $(-K_S)^\perp \subset \operatorname{Pic}(S)$ is non degenerate, even, of rank $g-d$ and its discriminant is (find it).
\end{exercise}

%\addcontentsline{toc}{section}{References}
%\printbibliography
%\clearpage
\end{document}
